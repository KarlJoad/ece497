\documentclass[10pt,letterpaper,final,twoside,notitlepage]{article}
\usepackage[margin=1.0in]{geometry} % 1 inch margins on all pages
\usepackage[utf8]{inputenc} % Define the input encoding
\usepackage[USenglish]{babel} % Define language used
\usepackage{amsmath,amsfonts,amssymb}
\usepackage{amsthm} % Gives us plain, definition, and remark to use in \theoremstyle{style}
\usepackage{mathtools} % Allow for text and math in align* environment.
\usepackage{thmtools}
\usepackage{thm-restate}
\usepackage{graphicx}

\usepackage[useregional]{datetime2}

\usepackage[
backend=biber,
style=numeric,
citestyle=numeric]{biblatex} % Must include citation somewhere in document to print bibliography
\usepackage{hyperref} % Generate hyperlinks to referenced items
\usepackage[nottoc]{tocbibind} % Prints the Reference/Bibliography in TOC as well
\usepackage[noabbrev,nameinlink]{cleveref} % Fancy cross-references in the document everywhere
\usepackage{nameref} % Can make references by name to places
\usepackage{caption} % Allows for greater control over captions in figure, algorithm, table, etc. environments
\usepackage{subcaption} % Allows for multiple figures in one Figure environment
\usepackage[binary-units=true]{siunitx} % Gives us ways to typeset units for stuff
\sisetup{per-mode=symbol}
\usepackage{csquotes} % Context-sensitive quotation facilities
\usepackage{enumitem} % Provides [noitemsep, nolistsep] for more compact lists
\usepackage{chngcntr} % Allows us to tamper with the counter a little more
\usepackage{empheq} % Allow boxing of equations in special math environments
\usepackage[x11names]{xcolor} % Gives access to coloring text in environments or just text, MUST be before tikz
\usepackage{tcolorbox} % Allows us to create boxes of various types for examples
\usepackage{ctable} % Greater control over tables and how they look
\usepackage{diagbox} % Allow us to have shared diagonal cells in tables
\usepackage{multirow} % Allow us to have a single cell in a table span multiple rows
\usepackage{titling} % Put document information throughout the document programmatically
\usepackage[linesnumbered,ruled,vlined]{algorithm2e} % Allows us to write algorithms in a nice style.
\usepackage{nth} % Programmtically give ordinal numbers.

\counterwithin{figure}{section}
\counterwithin{table}{section}
\counterwithin{equation}{section}
\counterwithin{algocf}{section}
\crefname{algocf}{algorithm}{algorithms}
\Crefname{algocf}{Algorithm}{Algorithms}

% Redefine the 'end of proof' symbol to be a black square, not blank
\renewcommand{\qedsymbol}{$\blacksquare$} % Change proofs to have black square at end

% Create a command for question-equation symbol
\newcommand{\qeq}{\ensuremath{\overset{?}{=}}}

% Common Mathematical Stuff
\newcommand{\Abs}[1]{\ensuremath{\lvert #1 \rvert}}
\newcommand{\DNE}{\ensuremath{\mathrm{DNE}}} % Used when limit of function Does Not Exist

% Vector Functions
\newcommand{\Magnitude}[1]{\ensuremath{\lVert #1 \rVert}}
\newcommand{\Proj}[2]{\ensuremath{\mathrm{proj}_{#2}(#1)}}
\newcommand{\Rej}[2]{\ensuremath{\mathrm{rej}_{#2}(#1)}}

% Math Operators that are useful to abstract the written math away to one spot
% Number Sets
\DeclareMathOperator{\RealNumbers}{\ensuremath{\mathbb{R}}}
\DeclareMathOperator{\AllIntegers}{\ensuremath{\mathbb{Z}}}
\DeclareMathOperator{\PositiveInts}{\ensuremath{\mathbb{Z}^{+}}}
\DeclareMathOperator{\NegativeInts}{\ensuremath{\mathbb{Z}^{-}}}
\DeclareMathOperator{\NaturalNumbers}{\ensuremath{\mathbb{N}}}
\DeclareMathOperator{\ComplexNumbers}{\ensuremath{\mathbb{C}}}
\DeclareMathOperator{\RationalNumbers}{\ensuremath{\mathbb{Q}}}

% Calculus operators
\DeclareMathOperator*{\argmax}{argmax} % Thin Space and subscripts are UNDER in display

% Signal and System Functions
\DeclareMathOperator{\UnitStep}{\mathcal{U}}
\DeclareMathOperator{\sinc}{sinc} % sinc(x) = (sin(pi x)/(pi x))

% Transformations
\DeclareMathOperator{\Lapl}{\mathcal{L}} % Declare a Laplace symbol to be used

% Logical Operators
\DeclareMathOperator{\XOR}{\oplus}

% These packages are more specific to certain documents, but will be availabe in the template
% \usepackage{esint} % Provides us with more types of integral symbols to use
% \usepackage[outputdir=./TeX_Output]{minted} % Allow us to nicely typeset 300+ programming languages
% \crefname{lstlisting}{listing}{listings}
% \Crefname{lstlisting}{Listing}{Listings}
% This document must be compiled with the -shell-escape flag if the packages above are uncommented

\graphicspath{{./Images/}} % Uncomment this to use pictures in this document
\addbibresource{../References.bib}

\begin{titlepage}
  \title{Chipyard: A RISC-V Development Framework}
  \author{Alexander Lukens \and Karl Hallsby \and Faculty Advisor: Dr.\ Jia Wang}
  % DTMdisplaydate{YYYY}{MM}{DD}{DoW}
  \date{\DTMdisplaydate{2021}{4}{9}{-1}} % We want to inform people when this document was last edited
\end{titlepage}

\begin{document}
\selectlanguage{USenglish}
\pagenumbering{gobble}
\maketitle
\pagenumbering{arabic} % 1,2,3 on content pages

\nocite{chipyard}

Alex Lukens and Karl Hallsby are undergraduate students researching Chipyard~\cite{chipyard} and RISC-V.
RISC-V is a open-source Reduced Instruction Set Computing~(RISC) instruction set architecture.
It is sponsored by many companies, and is under heavy research at University of California at Berkeley.
It was specifically designed to have a minimal core instruction set, that allows for string consutomization.
This allows RISC-V to scale as needed in appropriate situations.
Chipyard removes some of the manual aspects of designing a RISC-V CPU design, and instead moves the chip designer's responsibilities to determining the feature-set desired.
This promises to make development of these designs both quicker and more correct.
These promises also come at the price of Chipyard being under \emph{very} active development.
The API of Chipyard is not considered stable, and some features for working with FPGAs is not yet implemented.

RISC-V is a newer revision of older versions of earlier versions of the RISC instruction set\footnote{Not to be confused with the acronym of RISC~(Reduced Instruction Set Computing)}.
This revision defines a very small core set of integer-based instructions allowing for a very basic CPU to be constructed.
RISC-V also defines a variety of standardized extensions, enabling hardware support for floating-point arithmetic, atomic instructions, SIMD\footnote{Single Instruction Multiple Data} instructions, and mixed-multiply-divide instructions.
% TODO: Explain why this standardization is beneficial for RISC-V.

Chipyard is a framework for designing, building, testing, and refining RISC-V CPU designs.
It uses a combination of Verilog and Scala to provide its utilities.
The hardware is described using Verilog, and Scala is used to parameterize and automate the generation of these Verilog modules to build full-fledged processors.
To aid in testing, Chipyard also provided utilities to generate FPGA bitstreams, \texttt{mcs} files, among other data formats.

In this project we went in a variety of directions.
We started by building some of the already-defined CPU designs and creating custom CPU designs using Scala.
Then, we used the tools provided in Chipyard to test these designs against the RISC-V ISA standard, to ensure the generated designs were compliant with the RISC-V standard.
We moved onto testing the standard processor designs using an FPGA both as an accelerator, and as a general-purpose testing platform.
Lastly, we also uploaded a program written in a high-level language (C) to the FPGA, and ran it on the generated RISC-V CPU.\@

There are a variety of next steps that this work can go in.
The first is a re-design of the computer architecture classes here at IIT.\@
Because RISC-V is open-source and is relatively license-free, there are few legal concerns when it comes to CPU design and building.
This open nature is quite useful for tracking changes made to underlying CPU designs too.
Because RISC-V is currently being used in industry, there is strong documentation available for use.
Lastly, other universities already use Chipyard and RISC-V for computer architecture courses, meaning materials are already available.

From a research perspective, this project does require additional work to integrate the Arty FPGA with the Chipyard framework.
Another possible direction to take this is to design a GPIO device that allows for a single-step bus cycle, similar to the SANPER-1.
Lastly, a feature we did not have any time to investigate was the VLSI-generation workflow.

Overall, Chipyard provides many tools that greatly enhance and simplify the development of larger RISC-V CPU designs.
In addition, to make subsequent development easier, we are currently preparing a documentation manual that covers everything required to begin building and using Chipyard.
There do exist similar documentation materials on the Internet, but they are scattered and sometimes contradictory.
Preparing this documentation manual will allow us to curate and present the information required to set everything required up, and then move forward with the discussion.

% To make this print, you must include a citation somewhere in the document
\printbibliography[heading=bibintoc]{}
\end{document}

%%% Local Variables:
%%% mode: latex
%%% TeX-master: t
%%% End:
