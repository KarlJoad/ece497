% No need to add periods at the end of descriptions.
% LaTeX will add those for you. You only need to worry about periods in the middle
% of the description.
% DO NOT CAPITALIZE NORMAL GLOSSARY ENTRIES' name FIELD!!
% Instead, keep it lowercase, and use the \Gls command to have LaTeX uppercase
% the letter for you.

\newglossaryentry{fpgag}{
  name = {Field Programmable Gate Array},
  description = {An integrated circuit designed to be configured by a customer or a designer after manufacturing using software},
}

%\newacronym{fpga}{FPGA}{\Gls{FPGA}}
\newglossaryentry{fpga}{
  type = \acronymtype,
  name = {FPGA},
  description = {Field Programmable Gate Array},
  first = {FPGA~(Field Programmable Gate Array)\glsadd{fpgag}},
  see=[Glossary:]{fpgag}
}

% \newacronym{nic}{NIC}{Network Interface Card}
\newglossaryentry{nic}{
  type = \acronymtype,
  name = {NIC},
  description = {Network Interface Card},
}

\newglossaryentry{irg}{
  name = {Intermediate Representation},
  description = {The data structure or code used internally by a compiler or virtual machine to represent source code.
    An IR is designed to be conducive for further processing, such as optimization and translation},
}

\newglossaryentry{ir}{
  type = \acronymtype,
  name = {IR},
  description = {Intermediate Representation},
  first = {IR~(Intermediate Representation)\glsadd{irg}},
  see=[Glossary:]{irg}
}

\newglossaryentry{firrtlg}{
  name = {Flexible Intermediate Representation Register Transfer Language},
  description = {An \gls{ir} for digital circuits designed as a platform for writing circuit-level transformations},
}

\newglossaryentry{firrtl}{
  type = \acronymtype,
  name = {FIRRTL},
  description = {Flexible Intermediate Register Transfer Language},
  first = {FIRRTL~(Flexible Intermediate Register Transfer Language)\glsadd{firrtlg}},
  see=[Glossary:]{firrtlg}
}

\newglossaryentry{dslg}{
  name = {Domain-Specific Language},
  description = {A computer language specialized to a particular application domain.},
}

\newglossaryentry{dsl}{
  type = \acronymtype,
  name = {DSL},
  description = {Domain Specific Language},
  first = {\texttt{DSL}~(Domain-Specific Language)\glsadd{sbtg}},
  see=[Glossary:]{dslg}
}

\newglossaryentry{sbtg}{
  name = {Scala Build Tool},
  description = {sbt is an open-source Scala-based \Gls{dsl} to express parallel processing task graphs as a build tool for Scala and Java projects, similar to Apache's Maven and Ant},
}

\newglossaryentry{sbt}{
  type = \acronymtype,
  name = {\texttt{sbt}},
  description = {Scala Build Tool},
  first = {\texttt{sbt}~(Scala Build Tool)\glsadd{sbtg}},
  see=[Glossary:]{sbtg}
}


%%% Local Variables:
%%% mode: latex
%%% TeX-master: "doc"
%%% End:
