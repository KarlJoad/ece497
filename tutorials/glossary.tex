% No need to add periods at the end of descriptions.
% LaTeX will add those for you. You only need to worry about periods in the middle
% of the description.
% DO NOT CAPITALIZE NORMAL GLOSSARY ENTRIES' name FIELD!!
% Instead, keep it lowercase, and use the \Gls command to have LaTeX uppercase
% the letter for you.

\newglossaryentry{socg}{
  name = {System on a Chip},
  description = {An integrated circuit that integrates all or most components of a computer or other electronic system},
}

\newglossaryentry{soc}{
  type = \acronymtype,
  name = {SoC},
  description = {System on a Chip},
  first = {SoC~(System on a Chip)\glsadd{socg}},
  see=[Glossary:]{socg}
}

\newglossaryentry{fpgag}{
  name = {Field Programmable Gate Array},
  description = {An integrated circuit designed to be configured by a customer or a designer after manufacturing using software},
}

%\newacronym{fpga}{FPGA}{\Gls{FPGA}}
\newglossaryentry{fpga}{
  type = \acronymtype,
  name = {FPGA},
  description = {Field Programmable Gate Array},
  first = {FPGA~(Field Programmable Gate Array)\glsadd{fpgag}},
  see=[Glossary:]{fpgag}
}

% \newacronym{nic}{NIC}{Network Interface Card}
\newglossaryentry{nic}{
  type = \acronymtype,
  name = {NIC},
  description = {Network Interface Card},
}

\newglossaryentry{irg}{
  name = {Intermediate Representation},
  description = {The data structure or code used internally by a compiler or virtual machine to represent source code.
    An IR is designed to be conducive for further processing, such as optimization and translation},
}

\newglossaryentry{ir}{
  type = \acronymtype,
  name = {IR},
  description = {Intermediate Representation},
  first = {IR~(Intermediate Representation)\glsadd{irg}},
  see=[Glossary:]{irg}
}

\newglossaryentry{firrtlg}{
  name = {Flexible Intermediate Representation Register Transfer Language},
  description = {An \gls{ir} for digital circuits designed as a platform for writing circuit-level transformations},
}

\newglossaryentry{firrtl}{
  type = \acronymtype,
  name = {FIRRTL},
  description = {Flexible Intermediate Register Transfer Language},
  first = {FIRRTL~(Flexible Intermediate Register Transfer Language)\glsadd{firrtlg}},
  see=[Glossary:]{firrtlg}
}

\newglossaryentry{dslg}{
  name = {Domain-Specific Language},
  description = {A computer language specialized to a particular application domain.},
}

\newglossaryentry{dsl}{
  type = \acronymtype,
  name = {DSL},
  description = {Domain Specific Language},
  first = {\texttt{DSL}~(Domain-Specific Language)\glsadd{sbtg}},
  see=[Glossary:]{dslg}
}

\newglossaryentry{sbtg}{
  name = {Scala Build Tool},
  description = {sbt is an open-source Scala-based \Gls{dsl} to express parallel processing task graphs as a build tool for Scala and Java projects, similar to Apache's Maven and Ant},
}

\newglossaryentry{sbt}{
  type = \acronymtype,
  name = {\texttt{sbt}},
  description = {Scala Build Tool},
  first = {\texttt{sbt}~(Scala Build Tool)\glsadd{sbtg}},
  see=[Glossary:]{sbtg}
}

\newglossaryentry{soft-core}{
  name = {soft-core},
  description = {A CPU design (core) that is described using software, and is typically written out to an \gls{fpga}},
}

\newglossaryentry{simdg}{
  name = {Single Instruction Multiple Data},
  description = {Operations defined using a single instruction that takes multiple data values in simultaneously},
}

\newglossaryentry{simd}{
  type = \acronymtype,
  name = {SIMD},
  description = {Single Instruction Multiple Data},
  first = {SIMD~(Single Instruction Multiple Data)\glsadd{simdg}},
  see=[Glossary:]{simdg}
}

\newglossaryentry{mmiog}{
  name = {Memory-Mapped Input/Output},
  description = {Memory-mapped Input/Output uses the same address space to address both memory and I/O devices.
    The memory and registers of the I/O devices are mapped to (associated with) address values},
}

\newglossaryentry{mmio}{
  type = \acronymtype,
  name = {MMIO},
  description = {Memory-Mapped I/O},
  first = {MMIO~(Memory-Mapped Input/Output)\glsadd{mmiog}},
  see=[Glossary:]{mmiog}
}

\newglossaryentry{iotg}{
  name = {Internet of Things},
  description = {A dynamic global network infrastructure with self-configuring capabilities based on standard and interoperable communication protocols where physical and virtual ``things'' have identities, physical attributes, and virtual personalities and use intelligent interfaces, and are seamlessly integrated into the information network, often communicate data associated with users and their environments~\cite{iotBook}},
}

\newglossaryentry{iot}{
  type = \acronymtype,
  name = {IoT},
  description = {Internet of Things},
  first = {IoT~(Internet of Things)\glsadd{iotg}},
  see=[Glossary:]{iotg}
}

%%% Local Variables:
%%% mode: latex
%%% TeX-master: "doc"
%%% End:
