\chapter{Setup}\label{chap:Setup}
\section{Introduction}\label{sec:Introduction}
This document is intended to serve as a record of the work performed for the ECE 497 special project supervised by Professor Jia Wang during the Spring 2021 semester.
In this document, we will specify how our project repository was created, outline issues we ran into, and provide guidance on how to better setup the Chipyard Framework.

\section{Project Environment}\label{chap:Project_Environment}
The first step to using the Chipyard Framework is creating a project environment and obtaining all of the Chipyard dependencies.
In this document, we assume you are using Ubuntu 20.04 LTS, running in a virtual machine with at least:
\begin{itemize}
\item 4 cores
\item \SI{8}{\giga\byte} of RAM
\item \SI{250}{\giga\byte} disk image
\end{itemize}

Much of the disk space that has been allocated will be utilized, as the entire RISC-V toolchain and Xilinx Vivado suite require a large amount of disk space.

This document will work equally well in other distributions, so long as the versions of the dependencies are matched.
Chipyard also has explicit support for CentOS, which extends to Fedora and RHEL as well.
In addition, installing and using Linux natively works as well.

\section{Building Chipyard}\label{sec:Building_Chipyard}
Here, we present the necessary steps to retrieving all the dependencies required to set up Chipyard for local development and simulation use.
All of the code shown in the listings of this section is gathered in the \file{code} subdirectory.

\subsection{Chipyard Dependencies}\label{sec:Chipyard_Dependencies}
To gather the Chipyard dependencies, follow the \href{https://chipyard.readthedocs.io/en/latest/}{Chipyard} documentation closely.
Specifically, the \href{https://chipyard.readthedocs.io/en/latest/Chipyard-Basics/Initial-Repo-Setup.html}{Section 1.4} of the documentation outlines how to prepare your operating system for development using the Chipyard framework.

A paraphrased reproduction of these steps are shown below.

\subsubsection{Retrieve/Install Dependencies}\label{sec:Retrive_Install_Dependencies}
Chipyard relies on numerous dependencies and libraries to read files and build the required Verilog files.
In addition, Chipyard relies on \texttt{sbt}, the Scala Build Tool, as a majority of Chipyard and its dependencies are written in Scala.

\Cref{lst:Ubuntu_Chipyard_Deps_Setup} is a script that handles fetching and installing all the dependencies for you.
Note that this does \textbf{not} work for installing the dependencies for Linux distributions that do not use the \texttt{apt} package manager.

\begin{listing}[h!tbp]
\bashsourcefile{./code/ubuntu-chipyard-deps-setup.sh}
\caption{Fetch Chipyard Dependencies using \texttt{apt} on Ubuntu}
\label{lst:Ubuntu_Chipyard_Deps_Setup}
\end{listing}

The \href{https://www.xilinx.com/support/download.html}{Xilinx Vivado Suite} is important to be installed if any work regarding an FPGA is to be conducted.
In my case, I downloaded the ``offline installation'' version of the Xilinx Unified Installer (version 2020.2) so that the actually installation process will complete faster.
When conducting the installation, be sure to select ``Vitis'' instead of just selecting ``Vivado''.
Installing Vitis will install the complete Xilinx package, including Vivado, and is useful for implementing FPGA projects.

\section{Other Useful Projects}
\subsection{Freedom E SDK}
\hyperref{https://github.com/sifive/freedom-e-sdk}{This repository} is maintained by SiFive, and provides several useful tools for designing, uploading, and debugging software to FPGA devices.
This repository is specifically meant for use with SiFive IP, but can still be utilized for Chipyard projects with some modification.

\subsection{Freedom Tools}
\hyperref{https://github.com/sifive/freedom-tools}{This repository} is maintained by SiFive, and is used to generate several tools that will be used during this project, such as the GCC cross-compiler for RISC-V (and many extension sets of RISC-V), OpenOCD, which assists users in debugging their FPGA designs, RISC-V QEMU, and other useful software.
These tools take a considerable amount of time and disk space to compile so it is best to run the \texttt{make} command as \mintinline{bash}{make -j`nproc`} to parallelize compiling.

%%% Local Variables:
%%% mode: latex
%%% TeX-master: "../doc"
%%% End:
