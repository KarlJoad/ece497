\documentclass{weeklyslides}

\addbibresource{../References.bib}

\title[Weekly Report]{ECE 497: Special Project \\ Weekly Report}
\subtitle{Week 01}
\author{Alexander Lukens \and Karl Hallsby}
\institute{Illinois Institute of Technology}
\date{\today}

\begin{document}

\begin{frame}
  \titlepage{}
\end{frame}

\section{Work We Did}\label{sec:Work_We_Did}
\begin{frame}
  \frametitle{Software}
  \begin{itemize}
  \item Set up a git repository, remotely hosted on GitHub.
  \item Wrote basic \texttt{shell.nix} file to work with \href{https://github.com/target/lorri}{lorri} and \href{https://github.com/direnv/direnv}{direnv} to ensure environment consistency between machines.
    \begin{itemize}
    \item This \texttt{shell.nix} file can be converted so that
      we can build our project too, if we so wish.
    \end{itemize}
  \item Developed Beamer-derived \LaTeX{} document class for weekly slides.
  \item Read through several sections of ChipYard's documentation.
  \item Found and read ChipYard's IEEE paper.
  \end{itemize}
\end{frame}

\section{Existing Work}\label{sec:Existing_Work}
\subsection{\href{https://github.com/openhwgroup/cva6}{CORE-V CVA6}}\label{sec:CORE_V_CVA6}
\begin{frame}
  \frametitle{\nameref*{sec:CORE_V_CVA6}}
  \begin{itemize}
  \item Depends on verilator ($\geq$ 4.002, but not 4.106 or 4.108)
  \item Can use \texttt{spike} to resolve instructions to mnemonics
  \item Can run userland applications
  \item Supports emulation of the \href{https://reference.digilentinc.com/reference/programmable-logic/genesys-2/reference-manual}{Genesys 2}
    \begin{itemize}
    \item Xilinx FPGA
    \item Part Number XC7K325T-2FFG900C
    \end{itemize}
  \item Can flash to FPGA as well
  \end{itemize}

  Overall, this is a ``complete package'' for a \emph{single} RISC-V CPU.\@
  It has components and peripherals already designed and ready to be simulated/flashed to an FPGA.\@
\end{frame}

\subsection{ChipYard}\label{subsec:ChipYard}
\begin{frame}
  \frametitle{ChipYard~\cite{chipyard}}
  \begin{itemize}
  \item A platform for designing, simulating, and implementing RISC-V hardware systems using open-source libraries
  \item Supports three main RISC-V core designs:
    \begin{enumerate}
    \item Rocket Core
    \item BOOM (Berkeley Out-of-Order Machine)
    \item CVA6 Core
    \end{enumerate}
  \item It seems that the Rocket Core is the most uniformly supported (SiFive-blocks library provides several system components intended to be used with the Rocket Core)
  \item \href{https://github.com/sifive/fpga-shells}{\texttt{fpga-shells}} are used to reduce the number of wrappers for physical devices.
  \item FPGA prototyping is supported
    \begin{itemize}
    \item The specific FPGA board Alex already owns is explicitly supported (\href{https://www.xilinx.com/products/boards-and-kits/arty.html}{Xilinx Arty 35T}).
    \end{itemize}
  \end{itemize}
\end{frame}

\subsubsection{Toolchain}\label{subsubsec:Toolchain} %leaving this for you, Karl. Let me know if you need input
\begin{frame}
  \frametitle{\nameref{subsubsec:Toolchain}}
  \begin{enumerate}
  \item \href{https://github.com/riscv/riscv-tools}{\texttt{riscv-tools}}
    \begin{itemize}
    \item Includes compiler, assember, ISA simulator (\href{https://github.com/riscv/riscv-isa-sim}{\texttt{spike}}), Berkeley Boot Loader, and a proxy kernel.
    \item Many of these have been upstreamed to Linux, GNU binutils, GNU coreutils, etc.
    \item Included in ChipYard for consistency and usability.
    \end{itemize}
  \item \texttt{esp-tools}
    \begin{itemize}
    \item Fork of \texttt{riscv-tools}
    \item Intended to work with Hwacha non-standard RISC-V vector extension.
    \item Can demonstrate how to add additional RoCC accelerators to spike and higher-level toolchains.
    \end{itemize}
  \end{enumerate}
\end{frame}

\subsubsection{Simulation}\label{subsubsec:Simulation}
\begin{frame}
  \frametitle{\nameref{subsubsec:Simulation}}
  Supports several simulation platforms:
  \begin{description}
  \item[Verilator] An Open-Source platform for simulating RTL logic, maintained by \href{https://www.veripool.org/}{Veripool}.
  \item[FireSim] Intended for advanced simulation and is to be used on AWS cloud instances.
    Provides comprehensive I/O simulation, including timing-accurate DRAM, Ethernet, etc.\ simulations.
    Funding would be required if we intend to use this simulator.
  \end{description}
  \begin{itemize}
  \item We anticipate that the Xilinx Vivado suite should provide sufficient FPGA simulation for our needs.
  \item Provides instructions for complete simulation using Verilator, including waveform generation. ECE Department Saturn Server provides tools that can be used to analyze waveforms (CosmoScope)
  \end{itemize}
\end{frame}

\subsubsection{RTL Generators}\label{subsubsec:RTL Generators}
\begin{frame}
  \frametitle{\nameref{subsubsec:RTL Generators}}
  \begin{itemize}
  \item All generators are written in \href{https://www.chisel-lang.org/}{Chisel}.
  \item Given input parameters, Chisel outputs RTL-Verilog, which is then simulated.
  \item Some (if not all) generators can be mixed-and-matched, allowing for complete designs using separated components.
  \end{itemize}
\end{frame}


\subsubsection{Documentation}\label{subsubsec:Doc}
\begin{frame}
  \frametitle{\nameref{subsubsec:Doc}}
  \begin{itemize}
  \item The documentation for ChipYard is very thorough.
  \item Gets into enough detail for it to walk us through some steps.
  \end{itemize}
\end{frame}


\section{Our Goals}\label{sec:Our_Goals}
\begin{frame}
  \begin{itemize}
  \item Develop a working \href{https://riscv.org/}{RISC-V} chip prototype.
  \item Something else too.
  \end{itemize}
  \frametitle{Our Goals}

\end{frame}

\begin{frame}
  \frametitle{References}
  \printbibliography{}
\end{frame}

\end{document}

%%% Local Variables:
%%% mode: latex
%%% TeX-master: t
%%% End:
