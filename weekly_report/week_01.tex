\documentclass{weeklyslides}

\addbibresource{../References.bib}

\title[Weekly Report]{ECE 497: Special Project \\ Weekly Report}
\subtitle{Week 01}
\author{Alexander Lukens \and Karl Hallsby}
\institute{Illinois Institute of Technology}
\date{\today}

\begin{document}

\begin{frame}
  \titlepage{}
\end{frame}

\section{Existing Work}\label{sec:Existing Work}
\section{Work We Did}\label{sec:Work_We_Did}
\begin{frame}
  \frametitle{Software}
  \begin{itemize}
  \item Set up a git repository, remotely hosted on GitHub.
  \item Wrote basic \texttt{shell.nix} file to work with \href{https://github.com/target/lorri}{lorri} and \href{https://github.com/direnv/direnv}{direnv} to ensure environment consistency between machines.
    \begin{itemize}
    \item This \texttt{shell.nix} file can be converted so that
      we can build our project too, if we so wish.
    \end{itemize}
  \item Developed Beamer-derived \LaTeX{} document class for weekly slides.
  \item Read through several sections of ChipYard's documentation.
  \end{itemize}
\end{frame}

\subsection{ChipYard}\label{subsec:ChipYard}
\begin{frame}
  \frametitle{ChipYard~\cite{chipyard}}
  \begin{itemize}
  \item This is the intended behavior.
  \item A platform for designing, simulating, and implementing RISC-V hardware systems using open-source libraries
  \item Supports three main RISC-V core designs: Rocket Core, BOOM (Berkeley Out-of-Order Machine), and CVA6 Core
  \item It seems that the Rocket Core is the most uniformly supported (sifive-blocks library provides several system components intended to be used with the Rocket Core)
  \item It is important to note that FPGA prototyping is supported, and the specific FPGA board I already own is explicitly supported (\href{https://www.xilinx.com/products/boards-and-kits/arty.html}{Xilinx Arty 35T})
  \end{itemize}

\end{frame}

\subsubsection{Toolchain}\label{subsubsec:Toolchain} %leaving this for you, Karl. Let me know if you need input
\begin{frame}
  \frametitle{\nameref{subsubsec:Toolchain}}
\end{frame}

\subsubsection{Simulation}\label{subsubsec:Simulation}
\begin{frame}
  \frametitle{\nameref{subsubsec:Simulation}}
  \begin{itemize}
  \item Supports several simulation platforms, one of which is Verilator.
    Verilator is an Open-Source platform for simulating RTL logic, maintained by \href{https://www.veripool.org/}{Veripool}
  \item For advanced simulation, FireSim can be used to simulate fast FPGA boards to ensure pre-silicon verification and performance testing.
    FireSim is intended to be used on AWS cloud instances, so funding would be required if we intend to use this simulator.
    Provides comprehensive I/O simulation, including timing-accurate DRAM, Ethernet, etc.\ simulations.
    I anticipate that the Xilinx Vivado suite should provide sufficient FPGA simulation for our needs.
  \end{itemize}
\end{frame}

\subsubsection{Simulation}
\begin{frame}
  \frametitle{\nameref{subsubsec:Simulation}}
  \begin{itemize}
  \item Provides instructions for complete simulation using Verilator, including waveform generation. ECE Department Saturn Server provides tools that can be used to analyze waveforms (CosmoScope)
  \end{itemize}
\end{frame}


\subsubsection{RTL Generators}\label{subsubsec:RTL Generators}
\begin{frame}
  \frametitle{\nameref{subsubsec:RTL Generators}}
  \begin{itemize}
  \item
  \item
  \end{itemize}
\end{frame}


\subsubsection{Documentation}\label{subsubsec:Doc}
\begin{frame}
  \frametitle{\nameref{subsubsec:Doc}}
\end{frame}


\section{Our Focus}\label{sec:Our_Focus}
\begin{frame}
  \frametitle{Our Focus}
  \begin{itemize}
  \item Develop a working \href{https://riscv.org/}{RISC-V} chip prototype.
  \item Something else too.
  \end{itemize}

\end{frame}

\begin{frame}
  \frametitle{References}
  \printbibliography{}
\end{frame}

\end{document}

%%% Local Variables:
%%% mode: latex
%%% TeX-master: t
%%% End:
