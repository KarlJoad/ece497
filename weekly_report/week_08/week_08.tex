\documentclass{../weeklyslides}

\addbibresource{../../References.bib}

\title[Weekly Report]{ECE 497: Special Project \\ Weekly Report}
\subtitle{Week 07}
\author{Alexander Lukens \and Karl Hallsby}
\institute{Illinois Institute of Technology}
\date{\DTMdisplaydate{2021}{3}{18}{-1}}

\begin{document}

\nocite{chipyard}

\begin{frame}
  \titlepage{}
\end{frame}

\section{What We Did}
\begin{frame}
  \frametitle{What We Did}
  \begin{itemize}
  \item Olimex JTAG debugger came in, works correctly
  \item We can now finally upload C code to the FPGA and run it successfully
  \item At the moment, we are using the ``Sifive Freedom E300'' configuration (built on a previous version of Chipyard)
  \end{itemize}
  \begin{itemize}
  \item Continued to work on making our code a submodule.
  \item What happens is we want to extend chipyard while depending on chipyard.
  \item Leads to a circular dependency that SBT cannot resolve.
  \item Might be an unreasonable task for SBT.\@
    \begin{itemize}
    \item Or, Karl is writing the \texttt{build.sbt} incorrectly.
    \end{itemize}
  \end{itemize}
\end{frame}

\section{What We Learned}\label{sec:What_We_Learned}
\begin{frame}
  \frametitle{What We Learned}
  \begin{itemize}
  \item Chipyard Arty functionality is actively being worked on.
  \item Don't trust documentation, it is not always correct. (Sifive documents provide wrong UART speed, caused issues with UART communications)
  \item Making our code depend on Chipyard while also extending Chipyard's does not seem to be feasible.
  \end{itemize}
\end{frame}

\section{Next Steps}\label{sec:Next_Steps}
\begin{frame}
  \frametitle{Next Steps}
  \begin{itemize}
  \item Prepare full documentation of what we have learned so far.
    \begin{itemize}
    \item Debugger
    \item Chipyard and its build process
    \item Relevant helpful links
    \end{itemize}
  \item Identify reasonable goals in preparation for ECE Research Day.
  \end{itemize}
\end{frame}

\begin{frame}
  \frametitle{References}
  \printbibliography[heading=bibintoc]{}
\end{frame}

\end{document}

%%% Local Variables:
%%% mode: latex
%%% TeX-master: t
%%% End:
