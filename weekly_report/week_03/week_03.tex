\documentclass{../weeklyslides}

\addbibresource{../../References.bib}

\title[Weekly Report]{ECE 497: Special Project \\ Weekly Report}
\subtitle{Week 03}
\author{Alexander Lukens \and Karl Hallsby}
\institute{Illinois Institute of Technology}
\date{\DTMdisplaydate{2021}{2}{11}{-1}}

\begin{document}

\nocite{chipyard}

\begin{frame}
  \titlepage{}
\end{frame}

\section{What We Did}\label{sec:What_We_Did}
\begin{frame}
  \frametitle{\nameref*{sec:What_We_Did}}
  \begin{itemize}
  \item Attempted to flash default chip to Alex's FPGA.\@
  \item Generate a non-default chip
  \item Run the \texttt{asm} tests on the non-default chip
  \item Went spelunking through the repository to find options, their definitions, and their overrides.
  \end{itemize}
\end{frame}

\section{What We Learned}\label{sec:What_We_Learned}
\begin{frame}
  \frametitle{\nameref*{sec:What_We_Learned}}
  \begin{itemize}
  \item The repository is incredibly complicated
  \item \textbf{VERY} deep directory nesting (Partly due to Scala/Java project directory conventions).
  \item Putting the generated chip on an FPGA seems to be much more difficult than originally thought.
  \item Generating a non-default chip can be very easy or very hard.
    \begin{itemize}
    \item Some of the options that must be overriden to ensure a different chip is built and simulated/benchmarked are not easy to understand or find.
    \end{itemize}
  \end{itemize}
\end{frame}

\section{Next Steps}\label{sec:Next_Steps}
\begin{frame}
  \frametitle{\nameref*{sec:Next_Steps}}
  \begin{itemize}
  \item Continue trying to write the default chip out to Alex's FPGA and test.
  \item Practice generating other non-default chips to understand all the options used when generating a new chip.
  \item Hopefully, start defining a new, custom, chip using what we know, and building a \emph{very} small proof-of-concept.
  \end{itemize}
\end{frame}

\begin{frame}
  \frametitle{References}
  \printbibliography[heading=bibintoc]{}
\end{frame}

\end{document}

%%% Local Variables:
%%% mode: latex
%%% TeX-master: t
%%% End:
