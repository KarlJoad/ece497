% No need to add periods at the end of descriptions.
% LaTeX will add those for you. You only need to worry about periods in the middle
% of the description.
% DO NOT CAPITALIZE NORMAL GLOSSARY ENTRIES' name FIELD!!
% Instead, keep it lowercase, and use the \Gls command to have LaTeX uppercase
% the letter for you.

\newglossaryentry{socg}{
  name = {System on a Chip},
  description = {An integrated circuit that integrates all or most components of a computer or other electronic system},
}

\newglossaryentry{soc}{
  type = \acronymtype,
  name = {SoC},
  description = {System on a Chip},
  first = {SoC~(System on a Chip)\glsadd{socg}},
  see=[Glossary:]{socg}
}

\newglossaryentry{fpgag}{
  name = {Field Programmable Gate Array},
  description = {An integrated circuit designed to be configured by a customer or a designer after manufacturing using software},
}

%\newacronym{fpga}{FPGA}{\Gls{FPGA}}
\newglossaryentry{fpga}{
  type = \acronymtype,
  name = {FPGA},
  description = {Field Programmable Gate Array},
  first = {FPGA~(Field Programmable Gate Array)\glsadd{fpgag}},
  see=[Glossary:]{fpgag}
}

% \newacronym{nic}{NIC}{Network Interface Card}
\newglossaryentry{nic}{
  type = \acronymtype,
  name = {NIC},
  description = {Network Interface Card},
}

\newglossaryentry{irg}{
  name = {Intermediate Representation},
  description = {The data structure or code used internally by a compiler or virtual machine to represent source code.
    An IR is designed to be conducive for further processing, such as optimization and translation},
}

\newglossaryentry{ir}{
  type = \acronymtype,
  name = {IR},
  description = {Intermediate Representation},
  first = {IR~(Intermediate Representation)\glsadd{irg}},
  see=[Glossary:]{irg}
}

\newglossaryentry{firrtlg}{
  name = {Flexible Intermediate Representation Register Transfer Language},
  description = {An \gls{ir} for digital circuits designed as a platform for writing circuit-level transformations},
}

\newglossaryentry{firrtl}{
  type = \acronymtype,
  name = {FIRRTL},
  description = {Flexible Intermediate Representation Register Transfer Language},
  first = {FIRRTL~(Flexible Intermediate Representation Register Transfer Language)\glsadd{firrtlg}},
  see=[Glossary:]{firrtlg}
}

\newglossaryentry{rtlg}{
  name = {Register Transfer Language},
  description = {A type of object code a kind of intermediate representation that is very close to assembly language, such as that which is used in a compiler.
    It is used to describe data flow at the register-transfer level of an architecture},
}

\newglossaryentry{rtl}{
  type = \acronymtype,
  name = {RTL},
  description = {Register Transfer Language},
  first = {RTL~(Register Transfer Language)\glsadd{rtlg}},
  see=[Glossary:]{rtlg}
}

\newglossaryentry{dslg}{
  name = {Domain-Specific Language},
  description = {A computer language specialized to a particular application domain},
}

\newglossaryentry{dsl}{
  type = \acronymtype,
  name = {DSL},
  description = {Domain-Specific Language},
  first = {\texttt{DSL}~(Domain-Specific Language)\glsadd{dslg}},
  see=[Glossary:]{dslg}
}

\newglossaryentry{sbtg}{
  name = {Scala Build Tool},
  description = {sbt is an open-source Scala-based \Gls{dsl} to express parallel processing task graphs as a build tool for Scala and Java projects, similar to Apache's Maven and Ant},
}

\newglossaryentry{sbt}{
  type = \acronymtype,
  name = {\texttt{sbt}},
  description = {Scala Build Tool},
  first = {\texttt{sbt}~(Scala Build Tool)\glsadd{sbtg}},
  see=[Glossary:]{sbtg}
}

\newglossaryentry{softcore}{
  name = {Softcore},
  description = {A digital circuit design (typically a logic core) that can be wholly described and implemented using software and logic synthesis.
    Such a design is typically written out to an \gls{fpga}, but can be written out to other programmable logic devices},
}

\newglossaryentry{simdg}{
  name = {Single Instruction Multiple Data},
  description = {Operations defined using a single instruction that takes multiple data values in simultaneously},
}

\newglossaryentry{simd}{
  type = \acronymtype,
  name = {SIMD},
  description = {Single Instruction Multiple Data},
  first = {SIMD~(Single Instruction Multiple Data)\glsadd{simdg}},
  see=[Glossary:]{simdg}
}

\newglossaryentry{mmiog}{
  name = {Memory-Mapped Input/Output},
  description = {Memory-mapped Input/Output uses the same address space to address both memory and I/O devices.
    The memory and registers of the I/O devices are mapped to (associated with) address values},
}

\newglossaryentry{mmio}{
  type = \acronymtype,
  name = {MMIO},
  description = {Memory-Mapped I/O},
  first = {MMIO~(Memory-Mapped Input/Output)\glsadd{mmiog}},
  see=[Glossary:]{mmiog}
}

\newglossaryentry{iotg}{
  name = {Internet of Things},
  description = {A dynamic global network infrastructure with self-configuring capabilities based on standard and interoperable communication protocols where physical and virtual ``things'' have identities, physical attributes, and virtual personalities and use intelligent interfaces, and are seamlessly integrated into the information network, often communicate data associated with users and their environments~\cite{iotBook}},
}

\newglossaryentry{iot}{
  type = \acronymtype,
  name = {IoT},
  description = {Internet of Things},
  first = {IoT~(Internet of Things)\glsadd{iotg}},
  see=[Glossary:]{iotg}
}

\newglossaryentry{isag}{
  name = {Instruction Set Architecture},
  description = {An abstract model of a computer. It is also referred to as architecture or computer architecture.
    A realization of an ISA, such as a central processing unit (CPU), is called an implementation},
}

\newglossaryentry{isa}{
  type = \acronymtype,
  name = {ISA},
  description = {Instruction Set Architecture},
  first = {ISA~(Instruction Set Architecture)\glsadd{isag}},
  see=[Glossary:]{isag}
}

\newglossaryentry{uartg}{
  name = {Universal Asynchronous Receiver-Transmitter},
  description = {Computer hardware device for asynchronous serial communication in which the data format and transmission speeds are configurable},
}

\newglossaryentry{uart}{
  type = \acronymtype,
  name = {UART},
  description = {Universal Asynchronous Receiver-Transmitter},
  first = {UART~(Universal Asynchronous Reciever-Transmitter)\glsadd{uartg}},
  see=[Glossary:]{uartg}
}

\newglossaryentry{spig}{
  name = {Serial Peripheral Interface},
  description = {A synchronous serial communication interface specification used for short-distance communication, primarily in embedded systems},
}

\newglossaryentry{spi}{
  type = \acronymtype,
  name = {SPI},
  description = {Serial Peripheral Interface},
  first = {SPI~(Serial Peripheral Interface)\glsadd{spig}},
  see=[Glossary:]{spig}
}

\newglossaryentry{jvmg}{
  name = {Java Virtual Machine},
  description = {A virtual machine that enables a computer to run Java programs as well as programs written in other languages that are also compiled to Java bytecode},
}

\newglossaryentry{jvm}{
  type = \acronymtype,
  name = {JVM},
  description = {Java Virtual Machine},
  first = {JVM~(Java Virtual Machine)\glsadd{jvmg}},
  see=[Glossary:]{jvmg}
}

\newglossaryentry{elaboration}{
  name = {elaboration},
  description = {The build process of processor design generation.
    This involves finding all necessary submodules and ``gluing'' them together using the \texttt{TileLink} standard},
}

\newglossaryentry{riscv}{
  name = {RISC-V},
  description = {The fifth revision of an open-source \Gls{riscg} architecture, developed at University of California Berkeley},
}

\newglossaryentry{riscg}{
  name = {Reduced Instruction Set Computer},
  description = {A computer with a small, highly optimized set of instructions, rather than the more specialized set often found in other types of architecture},
}

\newglossaryentry{risc}{
  type = \acronymtype,
  name = {RISC},
  description = {Reduced Instruction Set Computer},
  first = {RISC~(Reduced Instruction Set Computer)\glsadd{riscg}},
  see=[Glossary:]{riscg}
}

\newglossaryentry{source}{
  name = {source},
  description = {Source code for a project.
    In the context of building software, building from source means compiling the project manually},
}

\newglossaryentry{man}{
  name = {\texttt{man}},
  description = {Command to fetch and open \texttt{man}ual pages from the system's informational database},
}

\newglossaryentry{multithread}{
  name = {Multi-thread},
  description = {The act of using multiple processes simultaneously, allowing for parallel computation},
}

\newglossaryentry{lazy_evaluation}{
  name = {lazy evaluation},
  description = {Computation model where expressions are evaluated as late as possible during program execution.
    This allows for infinitely recursive structures that do not cause program non-termination.
    Lazy evaluation tends to be most frequently used in functional programming languages, like \Gls{scala}},
}

\newglossaryentry{scala}{
  name = {Scala},
  description = {A strong statically typed general-purpose programming language which supports both object-oriented programming and functional programming.
    Designed to be concise, many of Scala's design decisions are aimed to address criticisms of Java},
}

\newglossaryentry{generator}{
  name = {Generator},
  description = {A singular, \glslink{parameterize}{parameterized} design that receives a number of parameters, and returns a number of objects (potentially one, or many) based on the provided information},
}

\newglossaryentry{extensible}{
  name = {Extensible},
  description = {An original product, built by someone else, can be \textbf{extended} to meet new requirements or to offer new functionality},
}

\newglossaryentry{parameterize}{
  name = {Parameterize},
  description = {The ability for an object to receive input parameters to change its behavior.
    This is the key functionality that allows a \gls{generator} to work},
}

\newglossaryentry{aws}{
  type = \acronymtype,
  name = {AWS},
  description = {Amazon Web Services},
  first = {AWS~(Amazon Web Services)},
}

\newglossaryentry{elfg}{
  name = {Executable and Linkable Format},
  description = {A common standard file format for executable files, object code, shared libraries, and core dumps},
}

\newglossaryentry{elf}{
  type = \acronymtype,
  name = {ELF},
  description = {Executable and Linkable Format},
  first = {ELF~(Executable and Linkable Format)\glsadd{elfg}},
  see=[Glossary:]{elfg}
}

\newglossaryentry{accelerator}{
  name = {accelerator},
  description = {A specialized processing unit that performs a single set of tasks very effectively.
    These can be thought of like \Glspl{dsl} for hardware.
    Some accelerators are domain-specific \Glspl{soc}, which are more like a regular CPU design, but are still not general-purpose compute units},
}

\newglossaryentry{assembly}{
  name = {assembly},
  description = {Any low-level programming language in which there is a very strong correspondence between the instructions in the language and the architecture's machine code instructions.
    Because assembly depends on the machine code instructions, every assembly language is designed for exactly one specific computer architecture.
    Assembly language may also be called symbolic machine code},
}

\newglossaryentry{ciscg}{
  name = {Complex Instruction Set Computer},
  description = {A computer in which single instructions can execute several low-level operations (such as a load from memory, an arithmetic operation, and a memory store) or are capable of multi-step operations or addressing modes within single instructions},
}

\newglossaryentry{cisc}{
  type = \acronymtype,
  name = {CISC},
  description = {Complex Instruction Set Computer},
  first = {CISC~(Complex Instruction Set Computer)\glsadd{ciscg}},
  see=[Glossary:]{ciscg}
}

\newglossaryentry{jtag}
{type = \acronymtype,
 name={JTAG},
 description={Joint Test Action Group},
 first = {JTAG~(Joint Test Action Group)\glsadd{jtagg}},
 see=[Glossary:]{jtagg}
 }
\newglossaryentry{jtagg}
{
 name={name},
 description={An industry standard connector type for verifying designs and testing printed circuit boards after manufacture. Used for communicating at a low level with the SoC design implemented in Chipyard}
 }



%%% Local Variables:
%%% mode: latex
%%% TeX-master: "doc"
%%% End:
