\chapter{Simulators}\label{chap:Simulators}
Chipyard currently has support for three simulators:
\begin{enumerate}
\item \nameref{sec:Verilator_Simulator}
\item \nameref{sec:VCS_Simulator}
\item \nameref{sec:FireSim_Simulator}
\end{enumerate}

Each of these are perfectly suitable for their task.
However, each one of these comes with its own benefits and drawbacks.
These will be discussed in their respective sections.

\section{Verilator}\label{sec:Verilator_Simulator}
\href{https://www.veripool.org/wiki/verilator}{Verilator} is an open-source (System)Verilog compiler and simulator.
Because it is open-source and written in a relatively high-level language (C/C++), it can be copmiled to any platform.
In addition, it offers sa wide range of functionality for Chipyard.
However, its biggest drawback right now is that it cannot perform validation simulations of procesor designs that would be implemented on a \gls{fpga}.

\section{VCS}\label{sec:VCS_Simulator}
\href{https://www.synopsys.com/verification/simulation/vcs.html}{VCS} is a closed-source proprietary (System)Verilog simulator and verifier.
It is trusted by some of the largest hardware design companies in the world, and is quite powerful.
It is currently used to simulate designs using all the available features of x86-based microprocessors.
In addition, VCS is the only Verilog simulator that can be used to simulate the Arty \Gls{fpga}.
Lastly, Arty support in VCS is still in active development, and is only on Chipyard's git \href{https://github.com/ucb-bar/chipyard/tree/arty-sim}{\texttt{arty-sim}} branch.

\section{FireSim}\label{sec:FireSim_Simulator}
\nocite{firesimPresentation}

%%% Local Variables:
%%% mode: latex
%%% TeX-master: "../doc"
%%% End:
