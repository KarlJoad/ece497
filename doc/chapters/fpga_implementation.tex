\chapter{FPGA Implementation}\label{chap:FPGA_Implementation}
This chapter is devoted to discussing how to implement a Chipyard-generated processor design on a \textbf{local} \gls{fpga} for quicker testing and general use.
Throughout the research project this manual was originally completed in, the Arty \gls{fpga} was used.
An image of the Arty board can be seen in \Cref{fig:Arty_FPGA}.
The Arty board is built using a Xilinx \gls{fpga} module and then Arty creates a board surrounding the particular chip.

\begin{figure}[h!tbp]
  \centering
  \includegraphics[scale=0.35]{./Arty_FPGA.png}
  \caption{Arty \Gls{fpga}}
  \label{fig:Arty_FPGA}
\end{figure}

\section{About}\label{sec:About}
The Chipyard Framework contains initial support for \gls{fpga} development and simulation of \gls{soc} designs.
At the moment this support is very limited, and is in active development.
As of \today, the best support for \Gls{fpga} Development for the Arty 35T \Gls{fpga} comes from a branch of Chipyard called \href{https://github.com/ucb-bar/chipyard/tree/arty-spi-flash}{arty-spi-flash}.
This branch fixes the \gls{uart} implementation and enables the \gls{spi} flash storage on the Arty \Gls{fpga} to allow users to store programs.

\section{Prerequisites}\label{sec:Prerequisites}
To assist with the proper setup, we approached the \Gls{fpga} implementation of an \Gls{soc} by following the \citetitle{FreedomDevGuide}~\cite{FreedomDevGuide}.
This outlined many of the steps we would eventually need to take, starting with purchasing an \href{https://www.digikey.com/en/products/detail/olimex-ltd/ARM-USB-TINY-H/3471388}{Olimex JTAG Debugger}~\cite{OlimexJTAG}.
Once the final image is flashed to the \Gls{fpga}, the debugger will allow the user to upload C programs and execute them on the RISC-V processor. Without the JTAG Debugger, we were unable to upload programs to the \Gls{fpga}, so this is a necessity.

\begin{figure}[h!tbp]
  \centering
  \includegraphics[width=0.7\linewidth]{./OlimexSetup.png}
  \caption{Olimex Debugger Setup~\cite[p.~5]{FreedomDevGuide}}
  \label{fig:olimexsetup}
\end{figure}

\section{Customizing an FPGA Image}\label{sec:Customizing}
In Chipyard, the directory used for all \Gls{fpga} prototyping functionality is \file{chipyard/fpga}, located in the root directory.
Inside this directory there are several important files.

\subsection{Makefile}\label{sec:Customizing_FPGA-Makefile}
Inside the Makefile is where you are able to define a custom subproject as shown in \Cref{subsec:Makefile_SUB_PROJECT}.
This allows users to control what files are compiled and generated for the \Gls{fpga} image.
This is highly recommended as it simplifies the workflow for repeated compilation attempts.

\subsection{Configuration Directory}\label{sec:Customizing_FPGA-Config_Directory}
The configuration directory for the Arty \Gls{fpga} is located under \file{chipyard/fpga/src/main/scala/arty/}.
This directory includes several useful files, including \file{Configs.scala}, \file{HarnessBinders.scala}, \file{IOBinders.scala}, and \file{TestHarness.scala}.

\subsubsection{\file{Configs.scala}}\label{sec:Customizing_FPGA-Configs.scala}
This file is where

\subsubsection{\file{HarnessBinders.scala}}\label{sec:Customizing_FPGA-HarnessBinders.scala}
This file is where

\subsubsection{\file{IOBinders.scala}}\label{sec:Customizing_FPGA-IOBinders.scala}
This file is where

\subsubsection{\file{TestHarness.scala}}\label{sec:Customizing_FPGA-TestHarness.scala}
This file is where

\subsection{Generated-src Directory}\label{sec:Customizing_FPGA-Generated-src_Directory}

\section{Generating the FPGA Image}\label{sec:Generating_FPGA_Image}
\subsection{Syntax}\label{sec:Generating_FPGA_Image-Syntax}


\section{Using the FPGA Image}\label{sec:Using_FPGA_Image}
\subsection{Flashing the Image}\label{sec:Flash_FPGA_Image}
\subsection{Uploading Programs to the FPGA}\label{sec:Upload_Programs_to_Flashed_FPGA}
%%% Local Variables:
%%% mode: latex
%%% TeX-master: "../doc"
%%% End:
