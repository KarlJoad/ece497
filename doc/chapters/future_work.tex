\chapter{Future Work}\label{chap:Future_Work}
In this chapter, we present opportunities that we feel can be leveraged due to the initial work we did during this research project.
We have chosen to break these opportunities up into several categories, each presented in a section below.

\section{Additional Research}\label{sec:Additional_Research}
Chipyard is a large and very complicated piece of software.
Simply getting over the initial hurdle of getting Chipyard to work and getting the generated \Gls{fpga} image written out comprised a majority of our work.
If we had more time to investigate Chipyard and become more familiar with its inner workings, we would like to further explore:
\begin{itemize}
\item Creating custom heterogenous CPU designs, elaborating them, and writing them out to an \Gls{fpga}.
\item Booting a minimal Linux kernel on the generated \gls{softcore}.
\end{itemize}

\section{Academic Applications}\label{sec:Academic_Applications}
\Gls{riscv} offers a lot to the academic world because it is a open-source CPU design.
This offers the chance to investigate the inner workings of the CPU and its implementation significantly more than many other architectures.
In fact, because \Gls{riscv} has been created by the \UCB{}, there are already academic materials available for use.

In the sections below, we discuss several courses offered at \IIT{} that, we believe, would be perfect candidates for a revamp using \Gls{riscv} and Chipyard.

\subsection{\href{http://bulletin.iit.edu/search/?P=ECE 242}{ECE 242 --- Digital Computers and Computing}}\label{sec:ECE_242}

\subsection{\href{http://bulletin.iit.edu/search/?P=ECE 441}{ECE 441 --- Microcomputers and Embedded Computing}}\label{sec:ECE_441}

\subsection{\href{http://bulletin.iit.edu/search/?P=ECE 485}{ECE 485 --- Computer Organization and Design}}\label{sec:ECE_485}

%%% Local Variables:
%%% mode: latex
%%% TeX-master: "../doc"
%%% End:
