\chapter{Future Work}\label{chap:Future_Work}
In this chapter, we present opportunities that we feel can be leveraged due to the initial work we did during this research project.
We have chosen to break these opportunities up into several categories, each presented in a section below.

\section{Additional Research}\label{sec:Additional_Research}
Chipyard is a large and very complicated piece of software.
Simply getting over the initial hurdle of getting Chipyard to work and getting the generated \Gls{fpga} image written out comprised a majority of our work.
If we had more time to investigate Chipyard and become more familiar with its inner workings, we would like to further explore:
\begin{itemize}
\item Creating custom heterogenous CPU designs, elaborating them, and writing them out to an \Gls{fpga}.
\item Booting a minimal Linux kernel on the generated \gls{softcore}.

\item Adding additional peripheral device functionality for the Arty and other \Gls{fpga} boards in Chipyard (buttons, LEDs, GPIO, etc.)

\item Writing a new C programming library for the Chipyard FPGA framework, and incorperating \Gls{jtag} debugger functionality directly into the framework.

\item Modeling performance of \gls{softcore} \Gls{riscv} designs versus discrete implementations.
\end{itemize}

\section{Academic Applications}\label{sec:Academic_Applications}
\Gls{riscv} offers a lot to the academic world because it is a open-source CPU design.
This offers the chance to investigate the inner workings of the CPU and its implementation significantly more than many other architectures.
In fact, because \Gls{riscv} has been created by the \UCB{}, there are already academic materials available for use.

In the sections below, we discuss several courses offered at \IIT{} that, we believe, would be perfect candidates for a revamp using \Gls{riscv} and Chipyard.

\subsection{\href{http://bulletin.iit.edu/search/?P=ECE 242}{ECE~242 --- Digital Computers and Computing}}\label{sec:ECE_242}
ECE~242 is intended to be an introductory course to some of the lowest levels of digital computing.
Namely, this involves in-depth discussion around both \Gls{cisc} and \Gls{risc} architectures, their differences, and how to write \gls{assembly}~\cite{iitECE242}.

Currently, the \Gls{risc} architecture taught is \href{https://en.wikipedia.org/wiki/MIPS_architecture}{MIPS}.
We feel that this is not going to help students in their future work, so we suggest teaching \Gls{riscv} instead.
The base instruction set is not much more complicated, with just forty-seven~(47) instructions for the user-level 32-bit integer instruction set~\cite[pp.~9--26]{riscvISASpec}.
Because the \Gls{riscv} \Gls{isa} is significantly more modern, many of the concepts learned here will still translate to other \Gls{risc} \Glspl{isa}.

By virtue of being more modern, students will gain an appreciation and knowledge of an architecture that they are far more likely to encounter in their career.
This is because \Gls{riscv} supports mixing and matching extensions, so the processors can actually be designed for anything from an embedded microcomputer to high performance computing.

\subsection{\href{http://bulletin.iit.edu/search/?P=ECE 441}{ECE~441 --- Microcomputers and Embedded Computing}}\label{sec:ECE_441}
ECE~441 teaches the concept of embedded computing in more depth.
It handles more advanced microprocessor features, such as hardware interrupting, memory design, and \Gls{mmio}~\cite{iitECE441}.

In previous iterations of this course, the SANPER-1 Educational Lab Unit was used.
This system is based on the \href{https://en.wikipedia.org/wiki/Motorola_68000}{Motorola 68000} series microprocessor.
Although there would be some work required to move from a \Gls{cisc} architecture to a \Gls{risc} one, we feel it is appropriate given how the world has already and will continue to move forward.
Although the principles in the MC68k are sound, they are also quite outdated.
We suggest a \Gls{riscv} CPU that implements the RV32E integer instruction set~\cite[p.~25]{riscvISASpec} because this course is particularly focused on embedded computing.

\Gls{riscv} processors can be built with the ability to have their bus cycles interrupted, which is a key feature of the SANPER.\@
In addition, highly desired features can be implemented in the \Gls{riscv} processor's hardware, by extending the already defined instructions with new ones.
This customization and flexibility is already making \Gls{riscv} a major competitor in the industrial world.
Taking this same flexible system and bringing it to academia will allow for further \Gls{riscv} environment maturation and more academically up-to-date graduates.

Taking this a step further, to keep the devices up-to-date, using an \Gls{fpga} might be appropriate as well.
This would allow for greater diversity in CPU design exposure during hte actual laboratory session.
If need be, key functionality can be added to the system between laboratory sessions.

\subsection{\href{http://bulletin.iit.edu/search/?P=ECE 485}{ECE~485 --- Computer Organization and Design}}\label{sec:ECE_485}
ECE~485 is designed to teach fundamental concepts of computer architecture, organization, and design~\cite{iitECE485}.
All of these topics can be covered and explored in even \emph{more} depth by having access to an \gls{extensible} \Gls{isa}.

Many of the sub-projects that Chipyard makes use of would be very appropriate for a reworked version of this course.
The \nameref{sec:cva6_Generator} and \nameref{sec:RISC-V_Sodor} would be perfect for this course.
They are already very small designs, implementing minimal functionality.

The \nameref{sec:RISC-V_Sodor} design would be best for introducing topics, because it has multiple stages that support different levels of simulation.
Because the simulations are done completely in software, there is minimal student overhead for testing new designs and learning about how the system is designed.
The different stages allow for students to view and simulate progressively more complex CPU designs.

The \nameref{sec:cva6_Generator} could be used as a simpler example of a full CPU design, as it supports multiple extensions and multiple privilege levels.
Because the \nameref{sec:cva6_Generator} is a single issue, in-order design, the circuitry is less complex than similar chips (\nameref{sec:Rocket_Chip} and \nameref{sec:BOOM_Generator}).
This makes the \nameref{sec:cva6_Generator} perfect to show component integration onto a single device.

We believe focusing the revamped version of this course around an \Gls{fpga} would also be most appropriate, as students could make new designs on-the-fly and test them.
This could open a completely different world up to this course.
Being able to not only learn about processor architecture and design, but the chance to implement this functionality on an \Gls{fpga} would make everything being learned tangible.

%%% Local Variables:
%%% mode: latex
%%% TeX-master: "../doc"
%%% End:
